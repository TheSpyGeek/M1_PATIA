% Created 2018-04-06 ven. 10:39
% Intended LaTeX compiler: pdflatex
\documentclass[11pt]{article}
\usepackage[utf8]{inputenc}
\usepackage[T1]{fontenc}
\usepackage{graphicx}
\usepackage{grffile}
\usepackage{longtable}
\usepackage{wrapfig}
\usepackage{rotating}
\usepackage[normalem]{ulem}
\usepackage{amsmath}
\usepackage{textcomp}
\usepackage{amssymb}
\usepackage{capt-of}
\usepackage{hyperref}
\author{TheSpyGeek}
\date{\today}
\title{}
\hypersetup{
 pdfauthor={TheSpyGeek},
 pdftitle={},
 pdfkeywords={},
 pdfsubject={},
 pdfcreator={Emacs 24.5.1 (Org mode 9.0.3)}, 
 pdflang={English}}
\begin{document}

\tableofcontents

\section{Planification Automatique - Krobot}
\label{sec:org6e342d7}

\subsection{Code de controle des moteurs}
\label{sec:org4705e8b}

   Nous avons réutilisé le code de Lavocat pour le controle des moteurs du robot.
Nous avons testé leur code pour voir ce qui était fait et on a lu une partie du code pour pouvoir
l'adapter à notre manière de dirigé le robot. Nous avons surtout étudié le fichier Controller.java
qui utilise la plupart des méthodes qui nous été necessaire.

\subsection{Définition du domaine et du problème}
\label{sec:orge1280cd}

Pour le domaine, nous avons défini une liste d'action que le robot peut éffectuer,
pour se déplacer, pour prendre un palet, pour relacher un palet\ldots{} 
Cet ensemble d'actions permettent de faire évoluer "l'état" d'un palet 
pour dans la finalité pour nous que tous les palets disponibles soient 
dans notre camp. Nous avons aussi créer une liste de prédicats qui 
traduisent l'état du jeu en cours, du nombre de palets encore disponible, 
de la position des palet, si le robot tient un palet dans ses pinces, 
si un palet est dans notre but ou pas.

Dans la définition du problème, nous avons choisi de définir la grille avec 12 noeuds (3 * 4) 
correspondant aux intersections des lignes de la table sans les intersections dans le but de l'équipe
adversse. Puis grace aux prédicats, on donne la liste de palet et sur quels noeuds sont les palets.
Dans le but on défini que tous les palets doivent être dans notre but (camp).

\subsubsection{Schéma du graphe correspondant à nos noeuds}
\label{sec:org64dc98f}
\begin{center}
\includegraphics[width=.9\linewidth]{domain_graph.png}
\end{center}

\subsection{Fonctionnement du code}
\label{sec:org365a186}

\subsubsection{Utilisation PDDL4J}
\label{sec:orge6a3802}

\subsubsection{Controle des moteurs pour se déplacer}
\label{sec:org1b6b06f}

\subsubsection{Intégration du controle des moteurs et PDDL4J}
\label{sec:orge344335}



\subsection{Amélioration possible}
\label{sec:orgd259dcf}

\begin{itemize}
\item grace aux équations de droites on pourrait savoir si le robot est allé trop loin si jamais le palet a bougé ou a été pris par un autre robot
\end{itemize}
\end{document}
